
\section*{Introduction}

\subsection*{Why we care about preferences}

Compared to other concepts in the study of international conflict and cooperation, state preferences have been understudied. Perhaps this is a consequence of black box theories of international relations, where it is assumed that all state preferences can be reduced to an automatic desire for more power. Alternatively, it may be because it is more difficult to measure something like preferences, as compared to more tangible material or institutional factors.

Yet this relative dearth of attention belies the importance of preferences in our theories of international processes. A number of formal theories of international relations require measures of preferences to be tested: the expected utility theory proffered by \citep{buenodemesquita:1983} has similarity of preferences as an important input, and attempts to expand studies of crisis bargaining to include coalitional dynamics \citep{wolford:2014}, mediation \citep{kydd:2003}, or the possibility of additional disputants \citep{gallop:2017} require a measure of state preferences in order to predict whether war will be the result of bargaining failure. Preferences have been used in empirical studies predicting bilateral trade, foreign aid, stability of international institutions and the incidence of conflict \citep{kastner:2007, derouen:heo:2004, stone:2004, gartzke:2007, braumoeller:2008}. 

One of the most important reasons we need a good measure of preferences is to correctly understand the democratic peace. It is difficult to entangle whether democracies avoid war with other democracies because of the intrinsic nature of democracy, or simply because they happen to have common ends. \citet{farber:gowa:1995} argued that democracies were only peaceful during the Cold War period because they had similar preferences and alliance structures. Similarly, \citet{gartzke:1998} argues that dissimilar preferences are a necessary condition for conflict. \citet{oneal:russett:1999e} responded by arguing that democracy has both a direct inhibiting effect on conflict, and an indirect one through influencing state preferences, though \citet{gartzke:2000} argued that even though democracies might have similar preferences, the residual of preferences from democracy explains conflict much better than the residual of democracy from preferences. While there has been some impressive development with our measures of preferences in recent years, a more accurate measure would really help us to disentangle the extent to which peace is the product of shared preferences, and the extent to which institutions and norms are driving peace.

While we would like an accurate measure of state preferences as an independent variable in our theory, measures of preferences can yield insights as a dependent variable in their own right. They could be used to see if the election of Donald Trump caused states to move their preference away from the United States, to see how the United States' preferences towards states in the Middle East changed after 9/11, or to see the impact of Russia's annexation of Crimea on their relations with their near-abroad states and the European Union.

[brief synopsis of method --- we need something relational, we need something that takes into account the variety of dimensions in which states interact with one another ... a topic that has received increasing attention in the networks literature in recent years relates to difficulties of working with arrays, see \citep{minhas:etal:2016} for some refs ]