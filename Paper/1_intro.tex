\setcounter{section}{1}

\section{Introduction}

\subsection{Why we care about preferences}

Compared to other concepts in the study of international conflict and cooperation, state preferences have been understudied. Perhaps this is a consequence of black box theories of international relations, where it is assumed that all state preferences can be reduced to an automatic desire for more power. Alternatively, it may be because it is more difficult to measure something like preferences, as compared to more tangible material or institutional factors.

Yet this relative dearth of attention should not be assumed to be caused by the relative unimportance of preferences in the study of international processes. A number of formal theories of international relations require measures of preferences to be tested: the expected utility theory proffered by \citep{buenodemesquita:1983} has similarity of preferences as an important input, and attempts to expand studies of crisis bargaining to include coalitional dynamics \citep{wolford:2014}, mediation \citep{kydd:2006}, or the possibility of additional disputants \citep{gallop:2017} require a measure of state preferences in order to predict whether war will be the result of bargaining failure. As \citet{hage:2011} has pointed out, preferences have been used in empirical studies predicting bilateral trade, foreign aid, stability of international institutions and the incidence of conflict \citep{kastner:2007, derouen:heo:2004, stone:2004, gartzke:2007, braumoeller:2008}. 

Preferences similarly pose the risk of omitted variable bias in a number of issues of paramount concern: without a good measure of preferences, it would be difficult to entangle whether democracies avoid war with other democracies because of the intrinsic nature of democracy, or simply because they happen to have common ends; similarly attempting to assess the impact of crisis behavior on a state's reputation requires us to determine how acceptable an outcome was to the state, as \citep{crescenzi:200X} does in his study of interdependence and resolve. Finally, measures of preferences can yield insights as an independent variable in their own right: they could be used to see if Edward Snowden's revelations about the United States spying caused states to move their preference away from the US's, to see how the United States's preferences towards states in the Middle East changed after 9/11, or to see the impact of Russia's annexation of Crimea on their relations with their satellite states and Western Europe.