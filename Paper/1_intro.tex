\section*{Why we care about preferences}

Compared to other concepts in the study of international conflict and cooperation, the foreign policy preferences of states have been understudied. %Knowing whether states have the capability to start a conflict (or engage in cooperation) only tells part of the story, we also need to know why they would want conflict (or cooperation). 
Perhaps this is a consequence of black box theories of international relations, where it is assumed that all state preferences can be reduced to an axiomatic desire for more power. Alternatively, it may be because it is more difficult to measure concepts such as preferences compared to tangible material or institutional factors.

This dearth of attention belies the importance of preferences in our theories of international processes. A number of formal theories of international relations require measures of preferences to be tested: the expected utility theory proffered by \citep{buenodemesquita:1983} has similarity of preferences as an important input. Further attempts to expand studies of crisis bargaining to include mediation \citep{kydd:2003}, coalitional dynamics \citep{wolford:2014}, or the possibility of additional disputants \citep{gallop:2017} require a measure of state preferences in order to predict whether war will be the result of bargaining failure. Preferences have been used in empirical studies predicting bilateral trade, foreign aid, stability of international institutions and the incidence of conflict \citep{derouen:heo:2004, stone:2004, gartzke:2007, kastner:2007, braumoeller:2008}. 

A substantive theoretical reason for why we need a good measure of preferences is to correctly understand the democratic peace. It is difficult to entangle whether democracies avoid war with other democracies because of the intrinsic nature of democracy, or simply because they appear to share similar ends. \citet{farber:gowa:1995} argue that democracies were only peaceful during the Cold War period because they had similar preferences and alliance structures. Similarly, \citet{gartzke:1998} argues that dissimilar preferences are a necessary condition for conflict. \citet{oneal:russett:1999e} respond by arguing that democracy has both a direct inhibiting effect on conflict, and an indirect one through influencing state preferences.\footnote{\citet{gartzke:2000} argued that even though democracies might have similar preferences, the residual of preferences from democracy explains conflict much better than the residual of democracy from preferences.} While there has been some impressive development with our measures of preferences in recent years, a more accurate measure is essential to disentangle the extent to which peace is the product of shared preferences, and the extent to which institutions and norms are driving peace.

%To provide this more accurate measure we take a multilayer network based approaches to estimating state preferences. Much of the extant literature has employed spatial weighting models \citep{signorino:ritter:1999,bailey:etal:2015} on relational data such as alliance behavior and UN voting scores. We make  

Much of the extant literature has focused on estimating state preferences by utilizing spatial weighting models on either alliance behavior or United Nations (UN) voting scores. These approaches have proven to be useful but there are two reasons to desire a different approach. First, alliances are rare and voting together in the UN is very common, so, by only focusing on the direct dyadic behavior, we risk mischaracterizing important relationships. Second, we would expect a better understanding of state preferences to help us predict state behavior, but as we show in figure \ref{perf:notus} adding measures of state preferences to a traditional model of interstate disputes yields relatively scant increases in our predictive ability.

\begin{figure}[ht]
	\centering
	\begin{tabular}{cc}
	\includegraphics[width=.5\textwidth]{roc_outSample_notUs.pdf} & 
	\includegraphics[width=.5\textwidth]{rocPr_outSampleNotUs.pdf}	
	\end{tabular}
	\caption{Assessments of out-of-sample predictive performance of Militarized Interstate Disputes using ROC curves and PR curves. AUC statistics are provided as well for both curves.}
	\label{fig:roc}
\end{figure}


We argue that we can improve on these measures of preferences using the same raw material by acknowledging that both alliance membership and UN voting are examples of relational data that takes place in a network. A bevy of research has shown that accounting for network structure necessitates an approach that can account for higher-order dependence patterns such as homophily and stochastic equivalence. As such, we make two contributions to the existing literature on state preferences. We introduce a latent factor model that accounts for higher-order dependence patterns across multiple layers. We show that our revised approach of measuring state preferences both better characterizes relationships that have counterintuitive results using existing measures of preferences, and this measure greatly enhances our ability to predict instances of conflict between countries when compared to existing measures.
