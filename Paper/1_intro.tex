\setcounter{section}{1}
\section{Introduction}
\subsection{Why we care about preferences}
Compared to other concepts in the study of international conflict and cooperation, state preferences have been understudied. Perhaps this is a consequence of black box theories of international relations, where it is assumed that all state preferences can be reduced to an automatic desire for more power. Alternatively, it may be because it is more difficult to measure something like preferences, as compared to more tangible material or institutional factors.

Yet this relative dearth of attention should not be assumed to be caused by the relative unimportance of preferences in the study of international processes. A number of formal theories of international relations require measures of preferences to be tested: the expected utility theory proffered by \citep{WarTrap} has similarity of preferences as an important input, and attempts to expand studies of crisis bargaining to include coalitional dynamics \citep{wolford:2014}, mediation \citep{kydd:year}, or the possibility of additional disputants \citep{gallop:2014} require a measure of state preferences in order to predict whether war will be the result of bargaining failure. As \citet{hage:2011} has pointed out, preferences have been used in empirical studies predicting bilateral trade, foreign aid, stability of international institutions and the incidence of conflict \citep{kastner:2007, derouen:heo:2004, stone:2004, gartzke:2007, braumoeller:2008}. Preferences similarly pose the risk of omitted variable bias in a number of issues of paramount concern: without a good measure of preferences, it would be difficult to entangle whether democracies avoid war with other democracies because of the intrinsic nature of democracy, or simply because they happen to have common ends; similarly attempting to assess the impact of crisis behavior on a state's reputation requires us to determine how acceptable an outcome was to the state, as \citep{crescenzi:200X} does in his study of interdependence and resolve. Finally, measures of preferences can yield insights as an independent variable in their own right: they could be used to see if Edward Snowden's revelations about the United States spying caused states to move their preference away from the US's, to see how the United States's preferences towards states in the Middle East changed after 9/11, or to see the impact of Russia's annexation of Crimea on their relations with their satellite states and Western Europe.

\subsection{Current measures of preferences: S-Scores}
The idea behind alliance portfolio measures is that we can infer a state's foreign policy by looking at the states they choose to align with. In the extreme case, if two states have all of the same allies, it is likely that their foreign policy goals are quite similar. Conversely, if all allies of one state are not allied to another, and vice versa, our best guess is that these states would have different aims and desires in foreign policy. \citet{altman:bdm:1979} encapsulate the logic when they note that "alliance commitments reflect a nation'sposition on major international issues." While the initial measures used to get at this intuition were Kendall's $\tau_{B}$, \citet{signorino:ritter:1999} convincingly pointed to flaws in this measure, notably its focus on rank-ordering as applied to a context where we instead care mostly about the presence or absence of an alliance. Thus, Signorino and Ritter introduce the S score, which has since been the most widely used alliance similarity measure.

The equation for the S score is:

\begin{equation}
S(P^i, P^j, W, L) = 1 - 2w_k \frac{d(P^i, P^j, W, L)}{d^{\text{max}}(W,L)}
\end{equation}

Where:
\begin{equation}
d(P^i, P^j, W, L) = \sum_{k = 1}^N \frac{w_k}{\Delta^\text{max}_{k}} |p^i_k - p^j_k|
\end{equation}

Where $w_k$ is the k'th element of a weight matrix, $d^\text{max}(W,L)$ is the maximal distance on a given dimension, and $\Delta$ is a normalizing constant. For the weight matrix, generally analysis has used S scores calculated with a weight matrix of ones--giving each potential ally equal weight--though the other plausible choice would be to weight states by import, for example using their share of world military capability, as calculated by \citet{cinc}.

One important distinction for these scores is that they are purely dyadic. One can look at the S-score between two states, but one cannot look at a state's preferences in comparison to a larger cluster, or note the movement a states preferences made over time. In monadic analysis, these score measures are not even available, and once we are dealing with situations involving more than two states, the number of S-scores necessary to fully characterize the preferences balloons quickly (it is the number of actors choose two).

These measures of alliance behavior also suffer from the relatively glacial movement and sparsity in these relationships. Formal alliances are relatively constant over time, whereas in many cases state preferences will be more fluid, and therefore these scores will be at best a lagging indicator of preferences. Furthermore, as \citet{hage:2011} points out, the fact that links are so rare creates artificial inflation of S scores. Signorino and Ritter propose a method to, somewhat account for this by weighting S scores using national capabilities, but these measures of capabilities are so skewed that this version of the measure is rarely used and would put undue emphasis on the alliances of the United States.