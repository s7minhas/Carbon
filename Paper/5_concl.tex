\section*{Conclusion}

We use a network methodology to create a new measure of state preferences using both UN Voting and alliance data. This measure of state preferences demonstrates face plausibility comparable or superior to existing measures when it comes to capturing the dynamics of a number of notable dyadic relationships.\footnote{There is an important limitation to keep in mind with the measure that we have generated, namely that we run separate models with a fixed three year rolling temporal window to estimate this measure. In future work, exploring the value-add from employing a single model that can generate time varying measures of state preference from multiple dimensions of state relations over time would likely be of notable value. A number of works have explored measuring low-dimensional relationships between actors using dynamic latent space models (e.g., \citealp{sewell:chen:2015}), and that work will certainly prove to be informative in future work that seeks to extend these ideas to the tensor setting.}  We then attempt to use this measure of state preferences in a predictive model of interstate disputes. In doing so, we find that the measure of preferences has the expected effect (states with similar preferences are less likely to be involved in disputes). Most importantly, a model of interstate conflict that includes our measure of preferences decisively outperforms models that include both of the most prominent existing measures of preferences.

While this measure of state preferences has yielded leverage in predicting conflict, it should also be useful in answering a number of other questions. One possible use would be to look at the measure of state preferences as an outcome variable, rather than a predictor. We could here look at how preferences change in tandem with leadership change -- can we find evidence, for example, that the election of Donald Trump moved the United States' foreign policy preferences away from the major Western European states and towards Russia's? We could also see how well this measure of state preferences predicts more collaborative behavior -- treaty membership or trade for example.

Ours is not the first paper to use network analysis to attempt to measure a theoretically important but difficult to observe phenomenon. For example, \citet{moody:white:2003} measured social cohesion by looking at the minimum number of paths connecting each actor in a group (and applied to high school friendship networks and among American businesses. \citet{beardsley:etal:2018} generates a measure of hierarchy by looking at a network of arms transfers and first finding the clusters in each group,  then how central an actor is within a given cluster, they go on to use this measure of hierarchy in a downstream model to explain interstate conflict, and similarly \citet{cranmer:etal:2015} use a measure of multiplex modularity in the network of alliances, trade, and international organization membership to measure ``Kantian fractionalization"--the extent to which the international system is fragmented between liberal and illiberal states.

In addition to helping us to understand state preferences, this sort of technique could help in measuring phenomena like social cohesion and hierarchy, or spatial contagion. If one wanted to use network approach to measure social cohesion, one could potentially do this with a Latent Class Model as in \citet{airoldi:etal:2008}. Using some measure of social interaction, this model would give the probability that each member is in a particular clique or subgroup in society. We could then look at how many members of society have probability of membership in multiple groups above a certain threshold to see how many individuals bridge different groups, contributing to our understanding of social cohesion. In understanding how phenomenon -- violence, diseases, new technology to name a few examples -- diffuse, many scholars rely on spatial lag models, which require the scholar to specify a spatial weights matrix that indicates how likely a phenomenon is to move from one area to another. Of course, we can never know the true adjacency matrix, and so scholars use a proxy, most often some form of geographic distance. We may be able to improve the predictive performance of spatial models by inferring a weights matrix using a network approach based on how the phenomenon has diffused in the past.

In general, there are a wide variety of substantive areas in which using network models to extract measurements of how actors relate may provide new insights. We would certainly not argue that the approach we present here can easily be plugged into alternative applications.\footnote{\citet{goldenberg:etal:2010} and \citet{kim:etal:2018} provide helpful discussions of various types of network models that have emerged to study cross-sectional and multilayer networks.} In fact, despite the benefits we highlighted of our approach for measuring state preferences, there are downsides. Most notably, our choice of manually choosing the last three years of data could certainly be iterated on in future research. \citet{park:sohn:2017}, for example, present a model to actually determine when there are discrete changes in the underlying group structure of network entities. In future research, further exploring how network models can be used to address measurement problems may prove to be a fruitful area of research.

% Regardless of the model used, if you are trying to use a latent network model for measurement, and especially if you are using the measure in a downstream model, you need to be aware of and account for uncertainty. To do this, we suggest running the downstream model repeatedly with different draws from the distribution for the latent space, then combining the models using \citet{rubin:1976}'s rules to obtain an estimate of the effect of your measure that accounts for uncertainty.
