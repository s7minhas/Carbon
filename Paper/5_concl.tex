\section*{Conclusion}

We use a network methodology to create a new measure of state preferences using both UN Voting data and alliance data. This measure of state preferences demonstrates face plausibility comparable or superior to existing measures when it comes to capturing the dynamics of a number of notable dyadic relationships. We then attempt to use this measure of state preferences in a predictive model of interstate disputes -- in doing so, we find that the measure of preferences has the expected effect (states with similar preferences are less likely to be involved in disputes) and inclusion of our measure leads to the measure of joint democracy becoming indistinguishable from $0$. Most excitingly, a model of interstate conflict that includes our measure of preferences decisively outperforms models that include both of the most prominent existing measures of preferences.

While this measure of state preferences has yielded leverage in predicting conflict, it should also be useful in answering a number of other questions. One possible use would be to look at the measure of state preferences as an outcome variable, rather than a predictor. We could here look at how preferences change in tandem with leadership change -- can we find evidence, for example, that the election of Donald Trump moved the United States' foreign policy preferences away from the major Western European states and towards Russia's? We could also see how well this measure of state preferences predicts more collaborative behavior -- treaty membership or trade for example.

Finally, this particular method allows us to control for confounding fixed effects. A closer examination of the relationship between democracy, preferences, and conflict could create a measure of the component of state preferences not explained by democracy.\footnote{We look at the entire measure of state preferences because our primary goal here is not to examine this relationship, but to create the most accurate measure of preferences possible.}  Similarly, the underlying methodology could be usefully applied to other questions. It could be used to estimate the cliques and alignments of elites in non-democratic states, or as a way of combining a number of related variables on development or democracy, in order to get an underlying latent measure. 

% While we would like an accurate measure of state preferences as an independent variable in our theory, measures of preferences can yield insights as a dependent variable in their own right. They could be used to see if the election of Donald Trump caused states to move their preference away from the United States, to see how the United States' preferences towards states in the Middle East changed after 9/11, or to see the impact of Russia's annexation of Crimea on their relations with their near-abroad states and the European Union.