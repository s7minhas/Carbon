\documentclass[12pt,pdflatex]{elsarticle}

%%%%%%%%%%%%%%%%%%%%%%%%%%%%%%%%%%%%%%%%%%%%%%%%%%
%%%%%%%%%%%%%%%%%%%% PREAMBLE %%%%%%%%%%%%%%%%%%%%
%%%%%%%%%%%%%%%%%%%%%%%%%%%%%%%%%%%%%%%%%%%%%%%%%%


% -------------------- defaults -------------------- %
% load lots o' packages

% references
\usepackage{natbib}

% Fonts
\usepackage[default,oldstyle,scale=0.95]{opensans}
\usepackage[T1]{fontenc}
\usepackage{ae}
% to colorize links in document. See color specification below
\usepackage[pdftex,hyperref,x11names]{xcolor}
% load the hyper-references package and set document info
\usepackage[pdftex]{hyperref}

% Generate some fake text
\usepackage{blindtext}

% layout control
\usepackage{geometry}
\geometry{verbose,tmargin=1.25in,bmargin=1.25in,lmargin=1.1in,rmargin=1.1in}
\usepackage{parallel}
\usepackage{parcolumns}
\usepackage{fancyhdr}

% math typesetting
\usepackage{array}
\usepackage{amsmath}
\usepackage{amssymb}
\usepackage{amsfonts}
\usepackage{relsize}
\usepackage{mathtools}
\usepackage{bm}
\usepackage[%
decimalsymbol=.,
digitsep=fullstop
]{siunitx}

% restricts float objects to be inserted before end of section
% creates float barriers
\usepackage[section]{placeins}

% tables
\usepackage{tabularx}
\usepackage{booktabs}
\usepackage{multicol}
\usepackage{multirow}
\usepackage{longtable}

% to adapt caption style
\usepackage[font={small},labelfont=bf]{caption}

% footnotes at bottom
\usepackage[bottom]{footmisc}

% to change enumeration symbols begin{enumerate}[(a)]
\usepackage{enumerate}

% to make enumerations and itemizations within paragraphs or
% lines. f.i. begin{inparaenum} for (a) is (b) and (c)
\usepackage{paralist}

% graphics stuff
\usepackage{subfigure}
\usepackage{graphicx}
\usepackage[space]{grffile} % allows us to specify directories that have spaces
\usepackage{placeins} % prevents floats from moving past a \FloatBarrier
\usepackage{tikz}
\usetikzlibrary{arrows, positioning, shapes.geometric,calc}
\usepackage{rotating}

% Spacing
\usepackage[doublespacing]{setspace}

% Add some colors
\definecolor{red1}{RGB}{253,219,199}
\definecolor{red2}{RGB}{244,165,130}
\definecolor{red3}{RGB}{178,24,43}

\definecolor{green1}{RGB}{229,245,224}
\definecolor{green2}{RGB}{161,217,155}
\definecolor{green3}{RGB}{49,163,84}

\definecolor{blue0}{RGB}{255,247,251}
\definecolor{blue1}{RGB}{222,235,247}
\definecolor{blue2}{RGB}{158,202,225}
\definecolor{blue3}{RGB}{49,130,189}
\definecolor{blue4}{RGB}{4,90,141}

\definecolor{purple1}{RGB}{191,211,230}
\definecolor{purple2}{RGB}{140,150,198}
\definecolor{purple3}{RGB}{140,107,177}

\definecolor{brown1}{RGB}{246,232,195}
\definecolor{brown2}{RGB}{223,194,125}
\definecolor{brown3}{RGB}{191,129,45}

\definecolor{cyan1}{RGB}{224,255,255}
\definecolor{cyan2}{RGB}{0,200,200}
\definecolor{cyan3}{RGB}{0,139,139}

% -------------------------------------------------- %


% -------------------- page template -------------------- %

\setlength{\headheight}{15pt}
\setlength{\headsep}{20pt}
\pagestyle{fancyplain}

\fancyhf{}

\lhead{\fancyplain{}{}}
\chead{\fancyplain{}{Networks \& State Preferences}}
\rhead{\fancyplain{}{}}
\rfoot{\fancyplain{}{\thepage}}

% ----------------------------------------------- %


% -------------------- customizations -------------------- %

% easy commands for number propers
\newcommand{\bl}[1]{{\mathbf #1}}
\newcommand{\first}{$1^{\text{st}}$}
\newcommand{\second}{$2^{\text{nd}}$}
\newcommand{\third}{$3^{\text{rd}}$}
\newcommand{\nth}[1]{${#1}^{\text{th}}$}

% easy command for boldface math symbols
\newcommand{\mbs}[1]{\boldsymbol{#1}}

% command for R package font
\newcommand{\pkg}[1]{{\fontseries{b}\selectfont #1}}

% approx iid
\newcommand\simiid{\stackrel{\mathclap{\normalfont\mbox{\tiny{iid}}}}{\sim}}

% -------------------------------------------------------- %

%%%%%%%%%%%%%%%%%%%%%%%%%%%%%%%%%%%%%%%%%%%%%%%%%%
%%%%%%%%%%%%%%%%%%%% DOCUMENT %%%%%%%%%%%%%%%%%%%%
%%%%%%%%%%%%%%%%%%%%%%%%%%%%%%%%%%%%%%%%%%%%%%%%%%

% remove silly elsevier preprint note
\makeatletter
\def\ps@pprintTitle{%
 \let\@oddhead\@empty
 \let\@evenhead\@empty
 \def\@oddfoot{}%
 \let\@evenfoot\@oddfoot}

\def\input@path{
	{C:/Users/Owner/Dropbox/Research/Carbon/Graphics/},
	{/Users/janus829/Dropbox/Research/Carbon/Graphics/},
	{/Users/s7m/Dropbox/Research/Carbon/Graphics/},
	{/Volumes/Samsung_X5/Dropbox/Research/Carbon/Graphics/},
	{/Users/maxgallop/Dropbox/Carbon/Graphics/}
	}

\graphicspath{
	{C:/Users/Owner/Dropbox/Research/Carbon/Graphics/},
	{/Users/janus829/Dropbox/Research/Carbon/Graphics/},
	{/Users/s7m/Dropbox/Research/Carbon/Graphics/},
	{/Volumes/Samsung_X5/Dropbox/Research/Carbon/Graphics/},
	{/Users/maxgallop/Dropbox/Carbon/Graphics/}
	}

\makeatother

\usepackage{etoolbox}
\makeatletter
    \patchcmd{\@author}{\global\let\@fnmark\@empty}{\global\let\@fnmark\@empty\global\let\@corref\@empty}{}{\@latex@error{Failed to patch \string\@author for \string\@corref reset}}
\makeatother

\journal{}

\begin{document}

% saying hello ----------------------------------------------- %
\thispagestyle{empty}
\begin{frontmatter}

\title{A Network Approach to Measuring State Preferences \tnoteref{t1}}
\tnotetext[label1]{Author order is alphabetical.}

% \tnotetext[t1]{Thanks to people.}

\author[strath]{Max Gallop}
\ead{max.gallop@strath.ac.uk}
\author[msu]{Shahryar Minhas\corref{cor1}}
\ead{minhassh@msu.edu}
%
\cortext[cor1]{Corresponding author}
%
\address[strath]{Departments of Political Science, University of Strathclyde}
\address[msu]{Department of Political Science, Michigan State University}

\begin{abstract}
\singlespacing{State preferences play an important role in international politics. Unfortunately, actually observing and measuring these preferences is impossible. In general, scholars have tried to infer preferences using either UN voting or alliance behavior. The two most notable measures of state preferences that have flowed from this research area are ideal points (Bailey et al., 2017) and S-scores (Signorino \& Ritter, 1999). The basis of both these models is a spatial weighting scheme that has proven useful but discounts higher-order effects that might be present in relational data structures such as UN voting and alliances. We begin by arguing that both alliances and UN voting are simply examples of the multiple layers upon which states interact with one another. To estimate a measure of state preferences, we utilize a tensor decomposition model that provides a reduced-rank approximation of the main patterns across the layers. Our new measure of preferences plausibly describes important state relations, and yields important insights on the relationship between preferences, democracy, and international conflict. Additionally, we show that a model of conflict using this measure of state preferences decisively outperforms models using extant measures when it comes to predicting conflict in an out-of-sample context.}
\end{abstract}

% journal abstract
% State preferences play an important, yet under discussed role in international politics. This is in large part because actually observing these preferences is impossible. Scholars have tried to infer preferences using UN voting or alliance behavior. Two notable measures of state preferences that have flowed from this research are ideal points and S Scores. These models use a spatial weighting scheme that discounts higher-order effects. We argue that UN voting and alliances are examples of the multiple layers upon which states interact with one another. To estimate state preferences from this multilayer structure, we introduce a latent factor model that provides a reduced-rank approximation of the main patterns across the layers. Our new measure of preferences plausibly describes important state relations. A model of conflict that uses this measure of state preferences decisively outperforms models using extant measures in predicting conflict in an out of sample context.

\end{frontmatter}
% ----------------------------------------------- %

\newpage\setcounter{page}{1}

\section*{Why we care about preferences}

% If we want to understand how two states interact in international politics, we need to know how similar their foreign policy preferences are.

Understanding state interactions in the realm of international politics necessitates some knowledge of states' foreign policy preferences vis-\`{a}-vis each other.  We can consider each state to have a preferred policy outcome on each possible issue that might arise in international relations. We can conceive of these preferred outcomes as an ideal point in multidimensional space. Given the fact that many issues in international politics are related, we can represent these preferences in far fewer dimensions than there are issues in international politics. These preferences will also change over time. When states have similar preferences on an issue, they will be more likely to collaborate to achieve their joint preference on that issue, and more generally states with similar foreign policy preferences will be cooperative in a larger proportion of their interactions. States with similar preferences are also unlikely to be involved in violent disputes with each other because the policy benefits of these disputes are unlikely to surpass the cost of fighting. Conversely, when states have highly dissimilar foreign policy preferences, cooperation will be difficult and violence will be more common. Unfortunately, while we have abundant data to measure the strength of states' economies, the volume of trade between states, or even their military power, it is much more difficult to measure states' preferences, because as with many social and political constructs they cannot be observed directly.

Effectively measuring state preferences would yield scholars a number of benefits. A number of formal theories of international relations require measures of preferences to be tested: the expected utility theory proffered by \citep{buenodemesquita:1983} has similarity of preferences as an important input. Further attempts to expand studies of crisis bargaining to include mediation \citep{kydd:2003}, coalitional dynamics \citep{wolford:2014}, or the possibility of additional disputants \citep{gallop:2017} require a measure of state preferences in order to predict whether war will be the result of bargaining failure. Preferences have been used in empirical studies predicting bilateral trade, foreign aid, stability of international institutions and the incidence of conflict \citep{derouen:heo:2004, stone:2004, gartzke:2007, kastner:2007, braumoeller:2008}. 

A substantive theoretical reason for why we need a good measure of preferences is to correctly understand the democratic peace. It is difficult to entangle whether democracies avoid war with other democracies because of the intrinsic nature of democracy, or simply because they appear to share similar ends. \citet{farber:gowa:1995} argue that democracies were only peaceful during the Cold War period because they had similar preferences and alliance structures. Similarly, \citet{gartzke:1998} argues that dissimilar preferences are a necessary condition for conflict. \citet{oneal:russett:1999e} respond by arguing that democracy has both a direct inhibiting effect on conflict, and an indirect one through influencing state preferences.\footnote{\citet{gartzke:2000} argued that even though democracies might have similar preferences, the residual of preferences from democracy explains conflict much better than the residual of democracy from preferences.} While there has been some impressive development with measures of preferences in recent years, a more accurate measure is essential to disentangle the extent to which peace is the product of shared preferences, and the extent to which institutions and norms are driving peace.

%To provide this more accurate measure we take a multilayer network based approaches to estimating state preferences. Much of the extant literature has employed spatial weighting models \citep{signorino:ritter:1999,bailey:etal:2015} on relational data such as alliance behavior and UN voting scores. We make  

Much of the extant literature has focused on estimating state preferences by utilizing spatial weighting models on either alliance behavior or United Nations (UN) voting scores. These approaches have proven to be useful but there are two reasons to desire a different approach. First, alliances are rare and voting together in the UN is very common, so, by only focusing on the direct dyadic behavior, we risk mischaracterizing important relationships. Second, we would expect a better understanding of state preferences to help us predict state behavior, but as we show in Figure \ref{fig:rocShitty} adding measures of state preferences to a traditional model of interstate disputes yields relatively scant increases in terms of predictive ability.

\begin{figure}[ht]
	\centering
	\begin{tabular}{cc}
	\includegraphics[width=.5\textwidth]{roc_outSample_noLatAngle.pdf} & 
	\includegraphics[width=.5\textwidth]{rocPr_outSample_noLatAngle.pdf}	
	\end{tabular}
	\caption{Assessments of out-of-sample predictive performance of Militarized Interstate Disputes using ROC curves and PR curves. AUC statistics are provided as well for both curves.}
	\label{fig:rocShitty}
\end{figure}

We can improve on these measures of preferences using the same raw material by acknowledging that both alliance membership and UN voting are layers of relationships that takes place simultaneously and in an interdependent context. Specifically, these relations between states constitute a multilayer network, in which the various layers correspond to different ways states are interacting with one another at a given time point. A bevy of research has shown that accounting for network structure necessitates an approach that can account for the indirect relations states share. As such, we make two contributions to the existing literature on state preferences. We utilize a multilinear tensor regression that enables us to measure how dependent actions of a particular dyad are across layers and time. We show that our revised approach of measuring state preferences both better characterizes relationships that have had counterintuitive results, and this measure greatly enhances our ability to predict instances of conflict.
\section*{Sources of Preference Measures: Alliance Portfolios and UN Voting}

Given that we cannot directly observe state preferences, scholars have attempted to estimate preferences using two main behavioral indicators: who states choose to ally with and how states vote at the UN. The idea behind alliance portfolio measures is that we can infer a state's foreign policy by looking at the states they choose to align with. In the extreme case, if two states have all of the same allies, it is likely that their foreign policy goals are quite similar. Conversely, if all allies of one state are not allied to another, and vice versa, our best guess is that these states would have different aims and desires in foreign policy. \citet{buenodemesquita:lalman:2008} encapsulate the logic by noting ``alliance commitments reflect a nation's position on major international issues''. Measures of alliance behavior do, however, suffer from the fact that these measures are largely static and sparsely occurring. Formal alliances are relatively constant over time, whereas in many cases state preferences will be more fluid, and therefore these scores will be at best a lagging indicator of preferences. Furthermore, \citet{hage:2011} shows that the rarity of links creates an artificial similarity of alliance portfolios.

We also have a relatively large corpus of behavioral information in UN Voting Records. The cost of voting in the UN is low, and so, scholars argue that measures of affinity based on UN voting are relatively representative of the underlying distribution of preferences \citep{gartzke:1998}. This is especially fortuitous because the methodology of inferring preferences from voting in a legislature is advanced. A few issues with these measures are that the potential benefit of winning UN votes is low, and so states might have incentives to vote against their preference as they are not costly signals, and the distribution of UN voting is prone to large supermajorities of the type rarely seen in ``ordinary" legislatures.

\subsection*{S-Scores \& Alliance Portfolios}

 One of the first measures used to measure preference similarity based on alliance portfolios is Kendall's $\tau_{B}$ \citep{buenodemesquita:lalman:2008}. This is operationalized as:

 \begin{equation}
	 \tau_{B} = \frac{n_{c} - n_{d}}{\sqrt{(n_{0} - n_{1})(n_{0} - n_{2})}}
 \end{equation}

 \noindent where $n_{c}$ is the number of pairs where both actor $i$ and $j$ have the same rank ordering (for example both the United Kingdom and the United States are more closely allied to Israel than to Iran), $n_{d}$ is the number of pairs where they have discordant rankings (the United States is more closely allied to Saudi Arabia than to Russia, Syria is more closely allied to Russia than to Saudi Arabia). The denominator attempts to adjust the total number of pairs with the number of ties: $n_{0}$ is the total number of pairs ($n(n-1)/2$), $n_{1}, n_{2}$ are measures for ties in both $i$ and $j$'s rankings respectively.

\citet{signorino:ritter:1999} convincingly point to flaws in this measure, notably its focus on rank-ordering as applied to a context where we instead care mostly about the presence or absence of an alliance. Additionally, adding more strategically irrelevant states will create artificially high $\tau_{B}$ statistics. Thus, Signorino and Ritter introduce the S-score, which has since been the most widely used alliance similarity measure.\footnote{\citet{bennett:rupert:2003} also find a stronger relationship between theoretical predictions and results when using S-scores than when using $\tau_{B}$.} The equation for the S-score is:

\begin{equation}
	S(P^i, P^j, W, L) = 1 - 2w_k \frac{d(P^i, P^j, W, L)}{d^{\text{max}}(W,L)}
\end{equation}

\noindent here $P^{i}$ and $P^{j}$ are vectors of length $N$, showing the relation that states $i$ and $j$ have to each of the $N$ states (including themselves).\footnote{The entries in $P^{i}$ could be binary, or they could be allowed to give more weight to stronger relationships, in the case of alliances they generally take on values between 0 and 3, with 0 denoting no alliance, and 1 an entente, 2 a non-aggression pact, and 3 a mutual defense pact.} $W$ is a vector of weights (normalized to sum to $1$), allowing a scholar to ascribe higher (lower) values for more (less) important states. For example, the weights matrix can be adjusted so that relations with China play a larger role in two states' similarity score than relations with a less globally relevant country. $d(P^i, P^j, W, L)$ is the distance between portfolio $i$ ($P^i$) and portfolio $j$ ($P^j$) based on the vector of weights $W$ and the scoring rule $L$. If using an absolute distance scoring rule, then $d(P^{i}, P^{j}, W, L) = \sum_{k = 1}^{N} \frac{w_{k}}{\Delta^{\text{max}_{k}}} |p^{i_{k}} - p^{j_{k}}|$.

In other words, we would calculate the distance between the two portfolios by looking at the absolute distance between each item in the portfolio ($p^i_{k}$ for state $i$'s relationship with state $k$) adjusted by the weight given to this relationship ($w_{k}$) and then normalized by $\Delta^{\text{max}_{k}}$, the maximum possible difference between $p^{i_{k}}$ and $p^{j_{k}}$. $d^{\text{max}}(W,L)$ is the largest distance between portfolios observed in the policy space, so this means that states at the maximum observed distance will have an S-score of $-1$, and those with identical portfolios will have an S-score of $1$. For the weight matrix, generally, analysts have used S-scores calculated with a weight matrix of ones--giving each potential ally equal weight--though the other plausible choice would be to weight states by importance, for example using their share of world military capability, as calculated by \citet{singer:small:1995}.

%An important point for these scores is that they are purely dyadic. One can look at the S-score between two states, but one cannot look at a state's preferences in comparison to a larger cluster, or note the movement in a states preferences over time. In monadic analysis, these score measures are not even available, and once we are dealing with situations involving more than two states, the number of S-scores necessary to fully characterize the preferences balloons quickly (it is the number of actors choose two).

In recent years, some scholars have started to argue that forming and maintaining alliances are not independent dyadic phenomena, but are rather part of an interdependent network (or series of networks). \citet{warren:2010} illustrates how extra-dyadic factors matter for alliance formation, in particular alliances with allies of your allies (and enemies of your enemies) are more likely. Further investigation uncovered that other network effects such as popularity and 2-stars can drive alliance formation as well \citep{cranmer:etal:2015}. The strategic nature of states when they form alliance ties leads to interdependence in the alliance network, as states seek to ensure their strength and security as efficiently as possible, privileging ties that complete triangles, and paying attention to the decreasing returns of additional alliances \citep{cranmer:etal:2012}. Later, \citet{warren:2016} demonstrated not only are network effects (like preferential attachment, density, and transitivity) important in the generation of the alliance network, but the alliance network and the conflict network are mutually constitutive. Similarly, \citet{kinne:bunte:2018} uncover links between the network of defense cooperation, and the network of bilateral loans, as states strategically loans to exert policy influence on other states, and strategically formed communities of defense cooperation in order to enhance the power and independence of its members.

\subsection*{Ideal Points \& UN Voting Similarity}

An advantage of utilizing UN General Assembly Voting, is that it allows the field to take advantage of methodological advances that have been made in the study of legislatures \citep{poole:rosenthal:1985}. \citet{bailey:etal:2015} do so by using an Item Response Theory model on UNGA voting. This model places states on a unidimensional latent preference space based on their voting behavior. The assumption of this model is that states' votes on a resolution are a function of states' ideal points, characteristics of the vote, and random error. In particular, for each bill $v$, a state's vote will be based on the latent variable $Z_{itv}$, representing the preference of state $i$, on vote $v$, at time $t$, defined such that:

\begin{equation}
	Z_{itv} = \beta_{v}\theta_{it} + \epsilon_{iv}
\end{equation}

\noindent Where $\beta_{v}$ is the discrimination factor of bill $v$, determining whether states with high ideal points will vote yes (if $\beta_{v}$ is negative) or no (positive). $\theta_{it}$ is state $i$'s ideal point at time $t$, and $\epsilon_{iv}$ is actor and bill specific error. Of course, we do not observe the latent variable, and only observe the outcome $Y_{itv}$, which will either be yes, no, or abstention. The model creates a pair of cutpoints $\gamma_{1v}$ and $\gamma_{2v}$ in the preference space, such that  if $Z_{itv} < \gamma_{1v}$, $Y_{itv} = \text{Yes}$, if $Z_{itv} > \gamma_{2v}, Y_{itv} = \text{No}$ and otherwise abstain. This allows the model to use observed vote outcomes in order to determine each states' ideal point in each year.

The authors specifically fix the parameters $\gamma_{1v}$ and $\gamma_{2v}$ such that the same bill will have the same value in different years, and they standardize and normalize $\theta$. They also use $\theta_{it-1}$ as a prior on $\theta_{it}$. With these constraints, they solve for the ideal points using a Metropolis Hastings Markov Chain Monte Carlo (MCMC).

Both methods relying on UN data, and those relying on alliances have difficulties distinguishing within '0's and '1's. For example, if we know two states are allies, we have reason to believe they have similar preferences, but if we know they are not allies, it is not clear whether they are enemies or they are indifferent -- the United States is ``not-allies" with both Bhutan and North Korea for example. As of 2012, using the Correlates of War projects alliance data, only about 1/8th of all dyads were between countries with any sort of alliance. Similarly, with UN voting, so many UN votes contain super-majorities and states vote together a huge proportion of the time. If we only look at yes and no votes in the UN general assembly, the median pair of states has voted together about 96\% of the time. If we include abstentions,  they have voted together 86\% of the time. So when two states vote together it is hard to distinguish between states voting together because of similar preferences, or just preferences that are not radically dissimilar. Both S-scores and Item Response theory succeed in adding granularity and nuance to these rough measures, but they are both limited by focusing only on a relationship between two states.

We can see the potential issues of focusing only on direct relations when we view how these two measures of preference treat relations over the Korean peninsula. If we take China, North Korea, South Korea, and the United States, we would expect the United States and South Korea to have preferences that are similar to each other and dissimilar from China and North Korea (and vice versa). Yet, if we look at extant measures of preferences (as of 2012), as depicted in Figure \ref{korean:prefs}, they do not seem to effectively characterize this relationship. S-scores based on alliance portfolios posit that China and the two Koreas are closely clustered, with the United States distant from all three, while ideal point distances put China's relationship with South Korea on par with its relationship with North Korea. Now it could be that these measures are producing a novel, counterintuitive, result, but given the failures of extant preference measures to add much to predictive models of conflict, one might be skeptical.

\begin{figure}[ht]
	\includegraphics[width=1\textwidth]{idPtScoreViz.pdf}
	\caption{Visualization of ideal point distance and S-score relationships between China (CHN), North Korea (DPRK), South Korea (ROK), and the United States (USA) in 2012. Within each panel a higher score denotes that the countries have a more positive relationship.}
	\label{korean:prefs}
\end{figure}

The sources of these surprising results become more clear when looking at the data on which they are based. In 2012, according to the Correlates of War, North Korea only had 3 alliances -- a non-aggression pact with the South, and alliances with Russia and China. Similarly, South Korea's only alliances were that non-aggression pact, and an alliance with the United States. Given the United States' many other allies, there was much more divergence between their alliance portfolio and South Korea's, than there was between the two Koreas' portfolios \citep{gibler:sarkees:2004}. Similarly, when looking at voting patterns at the UN, South Korea voted 50\% of the time with the US, 63\% of the time with North Korea, and 70\% of the time with China.

Even using this data, we can better approximate state preferences when we treat alliances and UN voting as relational data. When it comes to alliance behavior, the fact that North Korea was allies with China and Russia, and South Korea with the United States could give us additional information, because the alliance behaviors of countries such as the United States and China are so divergent. When looking at voting patterns at the UN, the need for a relational approach is even more apparent. While South Korea only voted with the US 50\% of the time at the UN, this was in the top 15\% of all countries in terms of voting with the US, whereas South Korea was in the bottom 20\% in terms of the proportion of time voting with China.

We thus argue that two changes can substantially improve measures of state preferences. First, both alliances and UN voting contain information about state foreign policy preferences, and given the limits of this information, we should find a way to use both. Second, by using network techniques and treating this data as relational data, we can wring more information from the stone, and get both a more nuanced, and more accurate view of state affinity and state preferences.

\section*{Synthesizing Measures of State Preference}

The approach that we introduce here to measure state preferences starts by assuming that both UN voting and alliance relationships are sources of information on how states relate to one another on the international stage. By accounting for the multiple layers upon which states interact with one another we can synthesize a better measure of state preferences than if we relied on any one measure alone. The idea of using multiple metrics to get a better handle on preferences is not new, in fact, Signorino and Ritter suggested it in introducing S-scores, which were designed to allow for aggregation of similarity on multiple dimensions (such as alliances and UN voting). The downside of this extant approach, however, is that it does not account for structural patterns that we often see in relational data.

Relational data is composed of observations between pairs of actors, or dyads. For both alliance relationships and UN voting, we are able to observe how the actors in the international system interact with one another across time. This system of interactions taken in its totality defines a network, and within these types of structures a bevy of research has shown that we need methods that go beyond assuming that interactions are taking place between just two actors in a vacuum \citep{wasserman:faust:1994,snijders:nowicki:1997,minhas:etal:2019}. As such we reformulate the problem of determining state preferences in terms of a network analysis. The goal of our approach is summarized in Figure~\ref{fig:tensViz}. In the top row, we represent UN voting and alliance patterns at time $t$ as a pair of adjacency matrices that form an evolving multiplex network.\footnote{The approach that we describe here can be generalized to a multiplex with more than two dimensions.} Our goal is to extract a lower dimensional representation of this system, such that the output is a series of $n \times n$ matrices, where $n$ represents the number of actors and in which the cross-sections denote our estimates of the preference similarities between countries.

\begin{figure}[ht]
	\centering
	\includegraphics[width=1\textwidth]{tensor_viz.png}
	\caption{The green and blue colors represent different relational measures and darker shading indicates later time periods. Our goal is to reduce the patterns found across those layers of relationships into a single measure.}
	\label{fig:tensViz}
\end{figure}

We generate these estimates through a multilinear tensor regression model that combines information across networks and time to measure how dependent the actions of a particular state are on another \citep{hoff:2015,minhas:etal:2016}. By incorporation information from multiple measures through a network perspective, we show that our approach improves on alternative measurements of state preference when it comes to predicting and explaining interstate conflict.

\subsection*{Multilinear Tensor Regression}

Our goal here is to measure a concept that is not directly observable. This is a problem that is very familiar in political science and a number of techniques based on measurement models have been developed to study political ideology \citep{martin:quinn:2002,konig:etal:2013}, human rights abuses \citep{fariss:2014}, and judicial independence \citep{linzer:staton:2015}. Obviously, the ideal points measure developed by \citet{bailey:etal:2015} also follows in this growing practice of extracting unobserved information via a spatial weighting scheme. Here we build on this general goal by developing a measurement of how a state relates to other states in a network context. Substantively, this goal is no different than how others have sought to find simpler representations of legislators and bills \citep{poole:rosenthal:1985,clinton:etal:2004}, but the key difference from shifting to the network is the recognition that we can better understand the relations between a pair of states by understanding how they relate to others in the international system.

We represent relational data that is longitudinal as a series of matrices $\{\bl Y_t : t = 1, \ldots T\}$, with each $\bl Y_t$ representing the actions between countries in a given year $t$. A cross-sectional entry such as $y_{i,j,v,t}$ from $\bl Y_t$ denotes an interaction that occurred between actors $i$ and $j$ at time $t$ across variable $v$. These series of matrices can be assembled in a tensor, $\bl Y$, which we use to represent actors interacting over time across multiple measures. Specifically, $\bl Y$ will have dimensions $n \times n \times v \times T$, where $n$ corresponds to the number of countries, $v$ to the number of variables (i.e., relational measures), and $T$ to the number of time periods. 
 
The information contained in this set of matrices can also be represented as a vector, where $(\mathbf y_t)$ represents a time series of vectors representing the dyadic interactions across multiple variables for $n$ countries through the period $t = 1, \ldots, T$. Transforming the data in this way enables us to think of what we are modeling in an explicitly time series context, where the evolution of multiple series is a function of their past behavior. Modeling such processes can often be accomplished using a vector autoregression (VAR) framework:

\begin{eqnarray}
	\bl y_t &=& { \Theta \bl y_{t-1} +\bl e_t}\\
	E[\bl e_t] &=& 0 \\
	E[\bl e_t \bl e^T_s] &=& \begin{cases}   \Sigma &\mbox{if } t=s, \\ 0 &\mbox{otherwise}\end{cases}
\end{eqnarray}

However, given the data requirements of estimating $\Theta$,\footnote{$\Theta$ has $n^4$ entries.} \citet{hoff:2015} employs a bilinear regression model such that the regression matrix is given by $ \Theta = \bl B \otimes \bl A$.\footnote{$\otimes$ is the Kronecker product.} As a result, we obtain the following basic specification: $\bl Y_t = \bl A \bl Y_{t-1}\bl B^T$. An alternative approach to this tensor decomposition could have involved the usage of a network latent variable model scheme. The key difference between those models and this tensor approach is that in the former we are projecting the relations between actors onto lower a dimensional space, where similarity in orientation indicates that actors are more likely to have similar dependence patterns. In the context of this model, $\bl A$ and $\bl B$ are $n \times n$ matrices of regressions coefficients, where each cross-sectional entry measures the level of dependence between the sending relationships of a particular actor and the $B$ the receiving relationships. 

The tensor regression problem can also be written somewhat more simply in tensor form, where it is clearer that we are essentially regressing the relational tensor $\bl Y_t$ from time $t$ on the tensor $\bl Y_{t-1}$ from time $t-1$:

\begin{equation}  
	\bl Y = \bl Y_{t-1} \boldsymbol{\times} \{ \bl A, \bl B \} + \bl E ,  
	\label{eqn:mltr}
\end{equation}

``$\boldsymbol{\times}$'' is a multilinear operator known as the ``Tucker product''. The Tucker product is used for higher-order singular value decomposition (SVD), the same way that matrix multiplication is used for matrix SVD \citep{kolda:bader:2009}. To illustrate how the Tucker product is calculated, say that we want to get the following expression: $\bl Y = \bl Y_{t-1} \boldsymbol{\times} \{ \bl A, \bl B\},$ where $\bl Y$ has dimensions $n_{1} \times n_{2} \times n_{3}$ and $\bl Y_{t-1}$ has dimensions $m_{1} \times m_{2} \times m_{3}$. The first step involves reshaping $\bl Y_{t-1}$ so that it is a matrix with  dimensions $m_{1} \times (m_{2} \times m_{3})$, then we multiply on the left by $\bl A$, next we reshape the result to an $n_{1} \times m_{2} \times m_{3}$ matrix.\footnote{Further details on this operator can be found in \citet{kolda:2006}.} This procedure would be applied iteratively among the remaining dimensions. We estimate this via Gibbs sampling using the procedure discussed in Hoff (2015).


The key parameters that we seek to estimate from this model are $\bl A$ and $\bl B$. These contain $n \times n$ regression coefficients and represent how previous actions along any of the $v$ parameters affect future interactions. The $ij$ coefficient in $\bl A$ measures the effect of an interaction from actor $j$ to actor $k$ along any of the $v$ parameters on the likelihood of an interaction from $i$ to $k$ in the next time period. Specifically, $a_{i,i'}$ capture how previous actions of $i'$ affect $i$ and $b_{j,j'}$ shows how actions that target $j$ are influenced by prior actions toward $j'$. If $a_{i,i'}$ is positive, this gives us a measure of how likely $i$ is to send an event to a third party $k$ given that $i'$ has already sent an event to $k$. More concretely, if we imagine that the event is alliance formation, then a positive $a_{i,i'}$ would indicate that $i$ and $i'$ are more likely to initiate alliances with the same third country. If $b_{j,j'}$ is positive on the other hand, this gives us a measure of how likely $j$ is to receive an alliance request from a third party $k$ given that $j'$ has already received that event from $k$. 
\subsection{Data Sources, Modeling choices}

We use the AME model on the two aforementioned measures of state amity to generate a combined measure of state preference similarity which accounts for network effects. We use the distance between states' ideal points (as calculated by \citet{voeten:XXXX} using UN data) and S-score for two states alliance portfolios. However, to facilitate comparison between the metrics, we first transform the S-score into a measure of distance between alliance portfolios.\footnote{D = 1 - S} We then standardize and normalize these two measures. This gives us an N by N by Y by 2 array, where the first two dimensions represent countries, the third dimension is the year, and the fourth is the particular measure of similarity. So the item at index (1,2,1,1) would be the transformed value of the S-score for countries XXX and YYY at the first year of our data (YYYY), similarly (1,2,1,2) would be the UN ideal point distance.

Another important question is the amount of temporal aggregation used. In our baseline model, we treat each year as separate and gain a unique observation of each states ideal point in each year. However, this raises a real risk of temporal inconsistency in the values. An alternative approach would be to have a rolling average for the measures of similarity over a number of years. This would allow us to infer a country's relative position not just by their behavior in a given year, but also their behavior in the past few years. The risk if we use too much temporal aggregation is that we are including data which is no longer relevant to a country's relative preferences. For instance, Turkey and Russia's relationship looks a lot more positive when we look at 2013 and 2014 then when we look at 2015. To that end, in addition to our baseline model where years are seen as independent, we also evaluate models where ... 

With this data, we run an AME model with a Gaussian link, and in particular we use the uDv term to estimate each states position in a two-dimensional latent space. We then evaluate whether their is additional utility gained from using this latent position, as compared to the component measures of similarity of alliance portfolio and UN ideal point distance.

\section{Constructing Latent Angle Measure}

\begin{figure}[ht]
	\centering
	\includegraphics[width=.7\textwidth]{latPlot}
	\caption{Latent factor plot.}
	\label{fig:latPlot}
\end{figure}

\subsection{Face Plausibility}

Are states close to who we'd expect them to be?

\subsection{Temporal Reliability}

Are our measures consistent?

\section{Model Competition}

\subsection{Data, Controls}

\subsection{In sample explanation}

Are coeffs starry and in right direction

\begin{figure}[ht]
	\centering
	\includegraphics[width=.7\textwidth]{betaEst}
	\caption{Parameter estimates from models with different measures of state preference. Point represents average estimate, line through the point represents the 95\% confidence interval.}
	\label{fig:coefP}
\end{figure}
\FloatBarrier

\subsection{Out of sample prediction}

\begin{figure}[ht]
	\centering
	\begin{tabular}{cc}
	\includegraphics[width=.5\textwidth]{roc_outSample} & 
	\includegraphics[width=.5\textwidth]{rocPr_outSample}	
	\end{tabular}
	\caption{Assessments of out-of-sample predictive performance using ROC curves and PR curves. AUC statistics are provided as well for both curves.}
	\label{fig:roc}
\end{figure}
\FloatBarrier

Are we right? Yes.

\section*{Conclusion}

We use a network methodology to create a new measure of state preferences using both UN Voting data and alliance data. This measure of state preferences demonstrates face plausibility comparable or superior to existing measures when it comes to capturing the dynamics of a number of notable dyadic relationships. We then attempt to use this measure of state preferences in a predictive model of interstate disputes -- in doing so, we find that the measure of preferences has the expected effect (states with similar preferences are less likely to be involved in disputes) and inclusion of our measure leads to the measure of joint democracy becoming indistinguishable from $0$. Most excitingly, a model of interstate conflict that includes our measure of preferences decisively outperforms models that include both of the most prominent existing measures of preferences.

While this measure of state preferences has yielded leverage in predicting conflict, it should also be useful in answering a number of other questions. One possible use would be to look at the measure of state preferences as an outcome variable, rather than a predictor. We could here look at how preferences change in tandem with leadership change -- can we find evidence, for example, that the election of Donald Trump moved the United States' foreign policy preferences away from the major Western European states and towards Russia's? We could also see how well this measure of state preferences predicts more collaborative behavior -- treaty membership or trade for example. Similarly, the underlying methodology could be usefully applied to other questions. It could be used to estimate the cliques and alignments of elites in non-democratic states, or as a way of combining a number of related variables on development or democracy, in order to get an underlying latent measure. 

\newpage

% Bib stuff
\clearpage
\bibliography{C:/Users/Owner/whistle/master}
% \bibliography{/Users/s7m/whistle/master}
\bibliographystyle{elsarticle-harv}\biboptions{authoryear}

\end{document}
