\documentclass[letterpaper]{article}
\usepackage{graphicx,fullpage}
\usepackage{hyperref}
\usepackage{geometry}
\usepackage[T1]{fontenc}
\usepackage[sc,osf]{mathpazo}

\geometry{
  body={6.5in, 8.5in},
  left=1.0in,
  top=1.25in
}
\usepackage{sectsty}
\sectionfont{\rmfamily\mdseries\Large}
\subsectionfont{\rmfamily\mdseries\itshape\large}
\setlength\parindent{0em}

% Make lists without bullets
\renewenvironment{itemize}{
  \begin{list}{}{
    \setlength{\leftmargin}{1.5em}
  }
}{
  \end{list}
}


\begin{document}
\thispagestyle{empty}

  
  
  \begin{minipage}{0.64\linewidth}
Max Gallop \\
Department of Government \& Public Policy \\
University of Strathclyde \\
16 Richmond Ave \\
G1 1XQ \\
Glasgow, UK
\end{minipage}
\begin{minipage}{0.45\linewidth}
  \begin{tabular}{lr}
    Email: & \href{mailto:max.gallop@strath.ac.uk}{\tt max.gallop@strath.ac.uk}  \\
    Website:& \href{https://www.strath.ac.uk/staff/gallopmaxmr/}{\tt strath/gallop}
  \end{tabular}
\end{minipage}
  
\vspace{1.5in}

{Editorial Team of Political Analysis via submission portal}

\vspace{0.5in}

Dear Colleagues:\\[1ex]

This letter accompanies our submission of a manuscript for your consideration. The manuscript ``A Latent Factor Approach to Measuring State Preferences'' develops a network based latent variable approach to measuring state preferences. We believe that this is a flexible framework that can be used to estimate unobservable concepts for a variety of phenomena political scientists are interested in studying.  \\[1ex]

In this paper, we focus on state preferences and provide a comparison of our approach to two prominent extant measures: S-scores and UN ideal point measures. We show that that our latent factor approach has better face validity than incumbent measures and that its expected effect in a regression on the initiation of militarized interstate disputes aligns with what scholars would expect. Most importantly, we show that by using our measure we can generate substantially more accurate out -of-sample prediction of MIDs than measures currently used in the literature. \\[1ex]

We look forward to your evaluation of this paper.\\[1ex]

Respectfully submitted,

\vspace{.1in}

The Authors

\vskip 0.5in
\hrule

\end{document}\bye
