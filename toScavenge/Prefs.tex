% July 30, 2012
% MDW penultimate version
% fixed word count, its is 8139 according to teashop
% moved figures and tables, arghh..

\documentclass[12pt,onesided,fullpage]{amsart}
\usepackage{amsfonts}
\usepackage{amsmath}
\usepackage{amssymb}
\usepackage{array,amsmath,graphicx,psfrag,amssymb,subfigure,tabularx,booktabs}
\usepackage{mparhack}
\usepackage{setspace}
\usepackage{natbib}
\usepackage{multicol}
\usepackage{dcolumn}
\usepackage{hyperref}
\usepackage{multirow}
\usepackage{color}
\usepackage[top=3cm, bottom=3cm, left=2.3cm, right=2.3cm]{geometry} 
%\pdfpagewidth=8.5in % for pdflatex
%\pdfpageheight=11in % for pdflatex
\begin{document}

\title{Measuring State Preferences}

\section{Why Care?}
Compared to other concepts in the study of international conflict and cooperation, state preferences have been understudied. Perhaps this is a consequence of black box theories of international relations, where it is assumed that all state preferences can be reduced to an automatic desire for more power. Alternatively, it may be because it is more difficult to measure something like preferences, as compared to more tangible material or institutional factors.

Yet this relative dearth of attention should not be assumed to be caused by the relative unimportance of preferences in the study of international processes. A number of formal theories of international relations require measures of preferences to be tested: the expected utility theory proffered by \citep{WarTrap} has similarity of preferences as an important input, and attempts to expand studies of crisis bargaining to include coalitional dynamics \citep{wolford:2014}, mediation \citep{kydd:year}, or the possibility of additional disputants \citep{gallop:2014} require a measure of state preferences in order to predict whether war will be the result of bargaining failure. As \citet{hage:2011} has pointed out, preferences have been used in empirical studies predicting bilateral trade, foreign aid, stability of international institutions and the incidence of conflict \citep{kastner:2007, derouen:heo:2004, stone:2004, gartzke:2007, braumoeller:2008}. Preferences similarly pose the risk of omitted variable bias in a number of issues of paramount concern: without a good measure of preferences, it would be difficult to entangle whether democracies avoid war with other democracies because of the intrinsic nature of democracy, or simply because they happen to have common ends; similarly attempting to assess the impact of crisis behavior on a state's reputation requires us to determine how acceptable an outcome was to the state, as \citep{crescenzi:200X} does in his study of interdependence and resolve. Finally, measures of preferences can yield insights as an independent variable in their own right: they could be used to see if Edward Snowden's revelations about the United States spying caused states to move their preference away from the US's, to see how the United States's preferences towards states in the Middle East changed after 9/11, or to see the impact of Russia's annexation of Crimea on their relations with their satellite states and Western Europe.

This paper seeks to evaluate the utility of different measures of preference on the study of international conflict. I discuss the incumbent measures of cooperation in the international relations literature--measures based on alliace networks and UN voting similarity. I then look at another possible measure of preferences inferred from event data on conflict and cooperation between states. The paper then ooks at the correlates of these measures in terms of other important factors in the conflict literature. I investigate how these measures fair at distinguishing states in conflict, states working together against a common foe, and states with little relation. I also see how these measures do at classifying states into plausible groups and identifying trends in particularly high leverage relationships. Finally, I investigate possible means of combining these measures to enhance their strengths and hide their weaknesses.

\section{Incumbent Measures}
In general, the proxies used for preferences in the international relations literature fall into two categories: those based on alliance data and those based on states' voting records in the United Nations.
\subsection{S-Score}
The idea behind alliance portfolio measures is that we can infer a state's foreign policy by looking at the states they choose to align with. In the extreme case, if two states have all of the same allies, it is likely that their foreign policy goals are quite similar. Conversely, if all allies of one state are not allied to another, and vice versa, our best guess is that these states would have different aims and desires in foreign policy. \citet{altman:bdm:1979} encapsulate the logic when they note that "alliance commitments reflect a nation'sposition on major international issues." While the initial measures used to get at this intuition were Kendall's $\tau_{B}$, \citet{signorino:ritter:1999} convincingly pointed to flaws in this measure, notably its focus on rank-ordering as applied to a context where we instead care mostly about the presence or absence of an alliance. Thus, Signorino and Ritter introduce the S score, which has since been the most widely used alliance similarity measure.

The equation for the S score is:

\begin{equation}
S(P^i, P^j, W, L) = 1 - 2w_k \frac{d(P^i, P^j, W, L)}{d^{\text{max}}(W,L)}
\end{equation}

Where:
\begin{equation}
d(P^i, P^j, W, L) = \sum_{k = 1}^N \frac{w_k}{\Delta^\text{max}_{k}} |p^i_k - p^j_k|
\end{equation}

Where $w_k$ is the k'th element of a weight matrix, $d^\text{max}(W,L)$ is the maximal distance on a given dimension, and $\Delta$ is a normalizing constant. For the weight matrix, generally analysis has used S scores calculated with a weight matrix of ones--giving each potential ally equal weight--though the other plausible choice would be to weight states by import, for example using their share of world military capability, as calculated by \citet{cinc}.

One important distinction for these scores is that they are purely dyadic. One can look at the S-score between two states, but one cannot look at a state's preferences in comparison to a larger cluster, or note the movement a states preferences made over time. In monadic analysis, these score measures are not even available, and once we are dealing with situations involving more than two states, the number of S-scores necessary to fully characterize the preferences balloons quickly (it is the number of actors choose two).

These measures of alliance behavior also suffer from the relatively glacial movement and sparsity in these relationships. Formal alliances are relatively constant over time, whereas in many cases state preferences will be more fluid, and therefore these scores will be at best a lagging indicator of preferences. Furthermore, as \citet{hage:2011} points out, the fact that links are so rare creates artificial inflation of S scores. Signorino and Ritter propose a method to, somewhat account for this by weighting S scores using national capabilities, but these measures of capabilities are so skewed that this version of the measure is rarely used and would put undue emphasis on the alliances of the United States.

\subsection{Structural Equivalence}
Structural equivalence, as used by \citet{maoz:etal:2006} is an alternative measure of the similarity of a set of associative measures. Structural equivalence, rather than looking at the distance between two states relationships between the states in the system, looks at the correlation between a pair of states relationships. This allows it to better capture higher order dependencies, whereas S (and $\tau_b$) only look at first order relationships.

\begin{equation}
SE_{ij} = \frac{\sum_{k = 1}^n (x_{ij} - \bar{x}_{.j})(x_{jk} - \bar{x}_{.j}) + \sum_{k=1}^n (x_{ki} - \bar{x}_{.i})(x_{kj} - \bar{x}_{j.})}{\sqrt{\sum_{k = 1}^n (x_{ik} - \bar{x}_{.i})^2 \sum_{k = 1}^n (x_{ki} - \bar{x}_{i.})^2}\sqrt{\sum_{k = 1}^n (x_{jk} - \bar{x}_{.j})^2 \sum_{k = 1}^n (x_{kj} - \bar{x}_{j.})^2}}
\end{equation}
\begin{itemize}
\item What is it?
\item Why bad?
\end{itemize}

\subsection{UN Voting}
An alternative way to measure preferences would be to look at behavioral indicators of these preferences. For most behavioral indicators this is impossible to do in a systematic measure as the contexts in which behaviors arrive differ between states, and it would be difficult to classify two behaviors as equivalent. However, we have a relatively large corpus of somewhat behavioral information in UN Voting Records. The cost of voting in the UN is low, and so, scholars have argued that measures of affinity based on UN voting are relatively representative of the underlying distribution of preferences \citep{gartzke:1998}. This is especially fortuitous because the methodology of inferring preferences from voting in a legislature is relatively advaced. \citet{voeten} classify different types of votes in the UN and use a model where both the type of vote, and states innate ideal points determine whether a state votes Yes, No, or Abstains on an issue. This spatial model leverages the fact that certain resolutions in the United Nations are voted on multiple times in order to identify situations where states change their ideal points.

Voeten and his coauthors use an MCMC algorithm to iteratively estimate both the ideal points of each state as well as the cutpoints which would cause states with a certain ideal point to vote for, against, or abstain. The model also incorporates prior years data into the estimation as the priors for the ideal points.

The issues here are that the voting behavior, especially in the EU general assembly, is not well behaved in the way that voting in the US Congress is. We actually can get some sense of it by the existence of multiple identical resolutions: UN resolutions have no legal force, and so most votes are symbolic. Thus UN voting is rife with nearly unanimous voting and other super-majorities, which means that the requirements to distinguish between state preferences are TALK MORE ABOUT WHY WE THINK THIS IS DUMB. ASK SCOTT. Another issue here, not necessarily with use of voting in general, but with this application, is the limitation to one dimension: it could be that two states which have very similar preferences on issues of trade---say the United States and [X]--might differ mightily on questions related to the Middle East, and in particular Israel.\footnote{You can see issues like this on legislative positions in the United States: during the post WW2 era, two democrats who might agree on the need to expand the social safety net might be diametrically opposed on issues of civil rights.} However, the dirth of contentious UN votes makes it difficult to add additional dimensions, and therefore ...


\section{Using Event Data}
An alternative proposal to get a useful proxy of state preferences is a more fluid and behavioral approach--using event data on cooperation and conflict to measure amity and enmity between states. Event data is (generally) machine coded data which uses news articles to capture the type and tenor of relationships between actors. The goal would be to take a measure of cooperative actions, and conflictual actions, and use these two counts to estimate each state's position in a two dimensional latent space in a given year. 

To estimate these ideal points, I turn to the event data collected by the International Crisis Early Warning Project (ICEWS).This data use mechanized learning processes to machine code millions of news stories in a number of languages, using TABARI and JABARI. In particular we, would seek out those stories where the actor doing something was some state, and the target of the action was another state. We would then use the verbs used in these stories to determine the valence of the action: in particular whether a state was acting cooperatively or conflictually towards its target. An example of a cooperative event would be [HERE], while a conflictual one would be [THERE]. Once we have a yearly count of each type of action, I take $\log\frac{\text{Total Cooperative Events} + 1}{\text{Total Conflictual Events + 1}}$. With this data, I estimate latent positions using a Generalized Bilinear Mixed Effects Model (hence GBME).

A Generalized Linear Mixed Effects Model is Generalized in that it uses a link function to account for data which does not have normal residuals, in this case the log transformation makes the residuals normal and we can use a Gaussian Link. It is a mixed effects model, in that it is hierarchical and hs both random intercepts (one for each member of a dyad) as well as potentially incorporating fixed effects. There is also a random effect for the dyad, which is particular we decompose into two components, $\gamma_{i,j} + x_{i,j}$ where $x$ is a bilinear term $z_i^Tz_j$, allowing this random effect to capture the higher order effects between the two members of the dyad (common neighbors and the like). This allows the model to capture 3rd order effects (transitivity,  balance, clusterability) not present in other latent space approaches.

To ensure some degree of temporal continuity, I include the previous year's bilinear effects as dyad specific covariates in the present year's estimation, following \citep{ward:ahlquist:rozenas:2013}. In addition, the results of the GBME estimation are in two-dimensional space, and so I use a procrustean transformation on the results to [EXPLAIN WHY HERE].

\subsection{Visualizing GBME Results}

\section{Visualizing Different Measures}
\subsection{Summaries of the Measures}
We thus have four competing measures of preferences. For each of these measures we have a vector of 10296 dyads\footnote{144 countries choose 2} and for each dyad there are 10 years. Of these we have the full data available for both alliance measures, whereas the UN voting data is missing 5054 observations, both due to countries not showing up in the dataset, and some states missing UN votes due to being invaded by the United States. The ICEWS data is missing 12510 of the 102960 observations, due to certain states having no observations int he event data corpus and thus being unable to be positioned in latent space. Summary statistics for each of the variables (except for the maximum, which are uniquely 1) are included in table \ref{summery}.

\begin{table}[ht]
\centering
\begin{tabular}{rrrrr}
  \hline
 & S & SEQ & UN & ICEWS \\ 
  \hline
min & -1.000 & -0.408 & -1.000 & -1.000 \\ 
  mean & 0.024 & 0.271 & 0.577 & 0.775 \\ 
  median & 0.239 & -0.088 & 0.666 & 0.853 \\ 
  sd & 0.418 & 0.344 & 0.347 & 0.24 \\ 
  temporal correlation & 0.996 & 0.997 & 0.974 & 0.898 \\ 
   \hline
\end{tabular}
\end{table}



One might also ask how these measures relate. This is summarized in table \ref{corr}, which looks at the correlations between the measures. The two alliance based measures, S and Structural Equivalence are, as we might expect, the most similar. Meanwhile, the measure based on event data is most distant from the other measures, as its closest relative is the other latent measure, using UN data, but the UN measues is as close or closer to those using alliance data. In addition, the ICEWS measure has a week negative relation to alliance based measures.

\begin{table}[ht]
\centering
\begin{tabular}{rrrrr}
  \hline
 & S & SEQ & UN & ICEWS \\ 
  \hline
S & 1.00 & 0.69 & 0.17 & -0.06 \\ 
  SEQ & 0.69 & 1.00 & 0.25 & -0.01 \\ 
  UN & 0.17 & 0.25 & 1.00 & 0.17 \\ 
  ICEWS & -0.06 & -0.01 & 0.17 & 1.00 \\ 
   \hline
\end{tabular}
\end{table}

\subsection{Correlates of Preferences}
An important question we might ask about these measures of preferences is how well they covary with the other variables of interest in the study of state preference in particular, and the study of cooperation and conflict more generally. There are two variables that seem especially likely to effect these measures of preference, one on the input side and the other on the output side. In terms of inputs, alliances, similar UN voting records, and events of conflict and cooperation are very likely to cluster geographically, and so one might wonder the effect of contiguity in particular, and proximity more generally on each measure. The other factor, which might produce particular preferences, especially as they relate to international politics, is state regime type.

States that are geographically proximate may have their amity overrepresented by these measures. If we assume two states have similar preferences, these similarities are more likely to result in alliances given proximity. Similarly, states have higher counts of cooperative (and conflictual) events given proximity, and so we might expect states who are similar to appear even more so if they are contiguous, and less so if not. To measure the effect of proximity, I use Weidman's CShape's program which provides a matrix of minimum distances among states. I code contiguity as 1 if states minimum distance is 0, else I code contiguity as a 0.

An important question raised by the dispute over democratic peace theory is whether democrats have similar preferences, or whether their institutions simply lead to less violent behavior in their relationships. We might thus want a measure of preference similarity that is not simply reproducing the similarity of regime types. For the measure of institutions, I follow the general trend by using the Polity IV index, and in particular create a measure for the absolute difference in the 21 point polity score between two states.

We thus have a question as to how our measures of preference correlate with these measures, and whether they provide any additional value. Ideally the measures would have the expected relationship with these correlates: difference in institutions is associated with difference in preferences, while geographic proximity is positively correlated with preference similarity. At the same time, we would not want too much of the variance in our measures of preferences explained by these variables, as that would remove much of their utiltiy. Table \ref{preference:dv} shows the results of regression of contiguity and insitutional difference on our measures. In each case, contiguity and institutional factors have a consistent and explicable relationship with the measure. The measures based on Structural Equivalence of Alliances and UN voting are the ones best explained by these covariates, and the measure based on ICEWS data and S scores of alliance portfolios are worst explained. For each variable,  greater difference in polity scores are associated with greater dissimilarity, and for all but the measure based on ICEWS data, proximity and similarity are linked. 


\begin{table}[!ht]
\caption{}
\label{} 
\begin{tabular}{ l D{.}{.}{2}D{.}{.}{2}D{.}{.}{2}D{.}{.}{2} } 
\hline 
  & \multicolumn{ 1 }{ c }{ S Score } & \multicolumn{ 1 }{ c }{Structural Equivalence } & \multicolumn{ 1 }{ c }{ UN Distance } & \multicolumn{ 1 }{ c }{ICEWS Distance } \\ \hline
 %             & Model 1 & Model 2 & Model 3 & Model 4\\ 
Polity Differene     & -0.01   & -0.01   & -0.02   & -0.00  \\ 
              & (0.00)  & (0.00)  & (0.00)  & (0.00) \\ 
Contiguous & 0.54    & 0.65    & 0.23    & -0.03  \\ 
              & (0.01)  & (0.01)  & (0.01)  & (0.01)  \\
 $N$           & 78626   & 78626   & 76919   & 78626  \\ 
$R^2$         & 0.05    & 0.14    & 0.09    & 0.00   \\ 
Resid. sd     & 0.40    & 0.31    & 0.34    & 0.25    \\ \hline
\end{tabular} 
 \end{table}

Returning to the relationship between measures of preference and conflict, including covariates based on contiguity and institutions has one major change on the performance of our variables. When we introduce these controls, in the model using UN voting data, insitutional variables have no discernible relationship with the incidence of disputes. One might interpret this as evidence that the ICEWS event data model has more value added, as it explains variance in the incidence of disputes that are not explained by widely used variables such as join democracy or geographic proximity, whereas much of the explanatory power of UN voting data overlaps with institutional measures. At the same time, the failure of S scores and measures of Structural Equivalence of alliance portfolios to explain which states become involved in conflict are thrown into starker relief. The association of similarity in these measures, and higher incidence of conflict do not appear to be caused by, for example, the fact that proximate states are more similar, and more likely to fight.


\begin{table}[!ht]
\caption{}
\label{} 
\begin{tabular}{ l D{.}{.}{2}D{.}{.}{2}D{.}{.}{2}D{.}{.}{2} } 
\hline 
  & \multicolumn{ 1 }{ c }{ Model 1 } & \multicolumn{ 1 }{ c }{ Model 2 } & \multicolumn{ 1 }{ c }{ Model 3 } & \multicolumn{ 1 }{ c }{ Model 4 } \\ \hline
 %             & Model 1  & Model 2  & Model 3  & Model 4 \\ 
(Intercept)   & -7.05    & -7.00    & -5.77    & -5.38   \\ 
              & (0.16)   & (0.14)   & (0.16)   & (0.16)  \\ 
S Score             & 0.67     &          &          &         \\ 
              & (0.19)   &          &          &         \\ 
Polity Difference     & 0.08     & 0.09     & 0.03     & 0.05    \\ 
              & (0.01)   & (0.01)   & (0.01)   & (0.01)  \\ 
Contiguous & 3.60     & 3.31     & 4.40     & 3.76    \\ 
              & (0.16)   & (0.17)   & (0.15)   & (0.13)  \\ 
Structural Equivalence           &          & 0.99     &          &         \\ 
              &          & (0.17)   &          &         \\ 
UN Distance           &          &          & -1.70    &         \\ 
              &          &          & (0.17)   &         \\ 
ICEWS Distance       &          &          &          & -1.77   \\ 
              &          &          &          & (0.14)   \\
 $N$           & 78626    & 78626    & 76919    & 78626   \\ 
AIC           & 3128.51  & 3110.55  & 2878.01  & 3009.75 \\ 
$\log L$     & -1548.26 & -1539.27 & -1423.01 & -1488.87 \\ \hline
 \multicolumn{5}{l}{\footnotesize{Standard errors in parentheses}}\\
\end{tabular} 
\caption{Predicting MIDs using Measures of Preference and Controls}
 \end{table}


\subsection{War, Peace, and Preferences}
For these measures of preferences, one important possible test is how well they identify which states are actually in conflict, compared to which states are not in conflict. For this, there is a relatively easy test: we determine if the scores for each of these measures of preferences are consistently lower for states who are enmeshed in a violent dispute than for those that are not. This is a relatively generous test: if a proxy is supposed to measure amity, or similarity of preferences, and two states are willing to shed blood and treasure because of their antipathy and dissimarity of preferences, those states should be coded as less similar than more peaceful dyads. To divide these dyads based on the presence of violenct conflict, we use Militarized Interstate Disputes (hence MIDs). MIDs are the current hegemonic measure of conflict in the international relations literature, and they are comprised of ``united historical cases of conflict in which the threat, display or use of military force short of war  by one member state is explicitly directed towards the government, official representatives, official forces, property, or territory of another state" \citep{mids}.

I run a simple t-test for each measure of preferences, comparing the level of the measure at those dyads enmeshed in a militarized interstate dispute and those where there is no MID. I also do the same but only looking at those cases involved in a fatal MID--those disputes where the action taken is either "Use of Force" or "War". The answers are revelatory: between 2001 and 2010, S-Scores and measures of Structural Equivalence are consistently \emph{higher} for dyads involved in both MIDs and Fatal MIDs. Now, if this only applied to MIDs, it would be less problematic, as we could conceivably explain the regularity as friends having low level disputes due to their large amounts of shared interests. Yet, the same is true of fatal MIDs--pairs of states that are at or near full scale war are classified as significantly more friendly by these measures. This is problematic. On the brighter side, both measures based on latent ideal  point estimation: the measure using UN voting data, and the measure using ICEWS event data are better able to differentiate between peace and war. In both cases MIDs and Fatal MIDs are associated with lower levels of amity, as we might expect. The confidence intervals for these t-tests are depicted in figure \ref{ttest}, and here the ICEWS event data measure performs somewhat better than the UN voting data, having a somewhat more positive point estimate, more precisely estimated.

\begin{figure}[htb]
\center
\includegraphics{TTests.pdf}
\caption{Confidence Intervals for T-Tests of MID incidence, or fatal MID incidence on the measures of preference.}
\label{ttest}
\end{figure}
\subsection{An example: 2002 and 2003}
Matrix of distance for US/Iran/France/UK/Russia/China/Iraq
\section{A few dyadic relationships over time}
The differing strengths and weaknesses of these measures can be illustrated by looking at a few particularly high profile cases. In the next section I  examine how each measure of preferences characterizes the relationship between: the United States and Israel, the US and China, Iraq and Iran, France and Germany, and the Koreas. These relationships differ on how dynamic the relationship is, as well as the valence of the relationship.

\subsection{Japan and China}
Japan and China have a complicated histrory and relationship. On the one hand, the neighboring states are large trading partners. On the other hand, they have a history of violent military conflict, with Japan brutally occupying China during the Second World War. The nations also have conflicting claims over the Diaoyu/Senkaku islands in the East China Sea.  In the early 2000s, there was a push in Japan to scale back economic developmental investment in China, both due to China's continuing growth and concerns about China's foreign and military policy. In addition, Japan's policies on Taiwan in 2005 (supporting both the Taiwanese Government and the United States) drew condemnation and protest. Beginning in the later half of the decade, there did appear to be some thaws in Sino-Japanese relations, punctuated by visits of Chinese President Hu Jintao to Tokyo and Japanese PM Aso Taro to Beijing.

\begin{figure}
\includegraphics{JapanChina.pdf}
\end{figure}

The potential measures of preference diverge sharply in their descriptions of the relationship between China and Japan. The S-Score of Alliance data characterizes the relationship as stable and very positive, whereas the structural equivalence of alliance data is stable and neutral. UN voting similarity is relatively stable and positive over the first decade of the 2000s, with an uptick after 2006 and a downtick after 2008. The measure based on ICEWS data is the most negative of the four, and it is consistently quite negative, with a decline in 2005 and 2006,  and then a larger increase between 2008 and 2010. In all likelihood, none of the measures correctly capture the tenor of the relationship, and it is likely a moderately negative relationship. The ICEWS data captures some of the temporal trends, yet often one year late.

\subsection{The United States and Russia}

The United States and Russia hae, since the fall of the Soviet Union had a complicated and fraught relationship, and one that has become both more relevant, and more strained recently due to Russia's military incursion into the Ukraine. In the time period covered, Russo-American affairs have had a generally negative tenor, due to the Bush administration's abrogation of the Anti-Ballistic Missile Treaty, the invasion of Iraq, and recent Russian aggression toward their neighbors. One particular moment that revealed the distance between the two states would be the Russian invasion of South Ossetia in 2008, and then the attempted ``reset" of US-Russian relations in 2009 leading to joint anti-terrorist exercises in 2010.  
\begin{figure}
\includegraphics{KoreasPrefs.pdf}
\end{figure}

The four measures all capture the generally negative tenor of the relationship. Measures based on alliance scores either never change, or only make a positive shift in the second half of the decade. The measure based on cooperative and conflictual ICEWS data captures the divergence in preferences peaking in 2008 and 2009 with the Russian attack in Georgia, and then the rebounding of relations following the ``reset." The UN voting measure has significantly less temporal variance until 2010, when the relationship appears to move from slightly negative to overwhelmingly positive, a shift that--despite the ``reset"--seems implausible given the current tenor of the relationship.

\subsection{US and Israel}
The United States and Israel have a very close, albeit complicated relationship. Some scholars have even sought to explain why their relationship was closer than old school theories of international relations might predict. The relation has only been strengthened during the Bush administration, as the US supported Israel's war in Lebanon, and was brought closer to Israel due to the terrorist attacks of 9/11 and invasions of Afghanistan and Iraq. Towards the end of the decade, the election of Barack Obama lead to some loosening of the relationship, which accelerated in recent years due to [REASONS].

\begin{figure}
\includegraphics{USIsrPref.pdf}
\end{figure}

The measures based on alliance voting understandably miss the relationship between the US and Israel, saying it is consistently neutral (for Structural Equivalence) or negative (for S scores). Meanwhile, the UN voting record, given the preponderance of votes where the United States and Israel alone are on one side of the issues (related to Israel) posits a very strong relationship. The ICEWS based measure finds somewhat of a middle-ground: positing a strong but inconsistent relationship. The notable inconsistency is a sharp and quickly reversed downturn in 2005. We might attribute that to pressure by the United States on Israel to scale back settlements[CITE SOME STUFF HERE] but it is likely a flaw in this measure of preferences.

\subsection{France and Germany}
France and Germany have, since the end of World War II had a relationship which has ranged from suspicious cooperation, to large scale economic integration. In the time period covered, France and Germany have had moments of joint opposition to a major ally's foreign policy preferences and they have continued their trend of major cooperation and integration. 

\begin{figure}
\includegraphics{FrGerPrefs.pdf}
\end{figure}

All four measures capture the generally positive tenor of the relationship. The measures based on alliance data and UN voting similarity correctly place France and Germany's relationships as among the closest in the world. The measures based onevent data are somewhat more moderate in classifying the relationship, which is probably to its detriment.

\subsection{Iran and Iraq}

The relationship between Iran and Iraq has been, to put it mildly, fraught. The states fought a devastating war of attrition in the 1980s, and sincehave been mutually suspicious. However, this relationship changed, due in large part to the acts of the United States. Following the 9/11 attacks, US President Bush grouped Iran and Iraq together in the ``axis of evil." Then, while the initial government the United States installed in Iraq was skeptical of Iranian interference, the more recent democratically elected governments have pushed for substantially closer ties with their Shi'a coreligionists in Iran.

\begin{figure}
\includegraphics{IRIQPref.pdf}
\end{figure}

The Iran/Iraq relationship in the 2000s shows the downside of the lack of dynamism in many measures of preference. The measure of structural equivalence posits that the states are neutral throughout the decade, while the S score posits a continuous positive relationship. The measure based on UN voting correctly places the relationship in the latter half of the decade as positive, but abstains from positing preferences in the first half of the decade. The measure based on ICEWS data shines here: it correctly points to increased amity due to the common US threat, then deteriorated relations when Iraq ws governed by a strong US client, and a gradual convergence of preferences as democratically elected Shia governments take over.
\section{Synthesis}
The different measures of preferences used all have their strengths, but also serious weaknesses. Alliance data has the most complete time series, but we on average see higher levels of these measures with states in military conflict than those outside of military conflict. Measures based on latent estimation of ideal points do better at classifying conflict and non-conflict, but both of these measures contain serious lacunae in the data: for ICEWS the missing data is time-invariant but more common, while for alliance data we see states drop out of the datasetat particular problematic moments--when they are in the middle of war or their government is changing. Most of the measure, for alliances and UN voting is remarkably consistent from year to year, with a correlation of over 0.99, while the measure for latent preferences based on ICEWS data varies too much over time (its temporal correlation is only 0.61. Ideally we would like something in the middle, though closer to 1 than to 0.6.

Where this leads us, as one could guess based on the section heading, is to ask whether there would be a way of combining these measures that woul accentuate their strengths and hide some of the weaknesses. I propose two such syntheses: one using Ensemble Bayesian Model Averaging (EBMA), and the other re-estimating the ideal points in a GBME, this time using the other measures of preferences as covariates.

\section{Landscape Analysis Applied}
The previous section has discussed how these dyadic measures of preferences compared to our expectations about certain dyadic relationships. Yet, recent methodological work has pointed to the profound weakness of dyadic analysis. In this section, I attempt to apply landscape theory, as introduced in \citep{axelrod:bennett:1992} to evaluate the coalitions predicted by these different measures of preferences. In particular, I look at how each measure of preferences would predict the coalitions in 2002 among the major Western nations, and a group of ``rogue states," notably Iran, Iraq and North Korea.

Landscape theory is an attempt to use a concept from physics, the total energy in the system, to capture each states satisfaction with a given set of alignments between two coalitions. Each states dissatisfaction is represented by:
\begin{equation}
F_i(X) = \sum_{j \neq i} s_j p_{ij} d_{ij} (X)
\end{equation}
where $s_j$ is the relative size or importance of state $j$, $p_{ij}$ is a measure of the preference similarity between $i$ and $j$, and $d_{ij}$ is the distance between $i$ and $j$ in alignment $X$, such that $d(X) = 1$ if they are in the same coalition, else $0$. We find the predicted coalitions of a set of countries by looking at the energy $E(X) = \sum_{i} s_i F_i(X)$. The local minima of that function will be the coalitions where frustration is minimized, and no country can do better by swapping coalitions. Axelrod and Bennett used landscape theory to investigate the groupings during World War II, and the end of the Cold War using relatively artisanal and refined measures of preference similarity. In this section, I attempt to see if the same theory, using the measures of preferences available can aptly predict the patterns of coalitions more recently.

The countries investigated in this analysis are the United States, Iraq, Iran, North Korea, Syria, Libya, the United Kingdom, Canada, France, Germany, Japan, Russia, China, and Israel. The measure of state size used, following Axelrod and Bennett, is the Correlates of War's Composite Index of National Capabilities (CINC). 

\section{Hegel Eat Your Heart Out}
Both the measures based on estimating ideal points in latent space, the one based on event data, and the one based on UN voting record, seem to be the strongest options as proxies for state preference. Yet each measure has strengths and weaknesses: the UN voting measure has somewhat less missing data, and more accurately assigns very high values to cases of constant cooperation. On the other hand, the ability of UN voting data to detect antipathy between states is less well shown, and the extreme levels of temporal stability make it sometimes unable to pick up shifts in preferences. Despite a higher mean level of amity, the ICEWS method appears better able to detect shifts, and to determine states with poor relationships. However, the data arguably has too much instability, and does less well at distinguishing countries with solid and uneventful relationships with those with extreme levels of cooperation. One possible way forward would be to leverage statistical techniques to find a sythesis between estimating ideal points using UN voting records, and doing so using ICEWS event data. I propose two ways to do so, and briefly loko at the plausibility of the results.

\subsection{GBME}
The first way to use event data and UN voting records in tandem would be to incorporate the ideal points from UN voting into the GBME estimation process to find ideal points, as dyad specific covariates. I thus rerun the GBME using a few covariates of interest

\begin{equation}
\end{equation}

where the $X_r$ and $X_s$ are the sender and receiver's polity scores, and $X_d$ contains both the lagged bilinear term, (as discussed in the previous section) and the UN data. 


\subsection{EBMA}
An alternative would be to use the measures of preferences here, as well as covariates of note, and leaving the models intact, combine models using Ensemble Bayesian Model averaging. This provides a principled method to use multiple models, leveraging each model's ability to correctly predict some cases, and covering for the weaknesses of others. The EBMA method also provides us another tool for evaluating the different measures of preferences, as it assigns differential weight to each measure, and those measures with more weight could be considered more reliable measures of preferences.

\subsection{Comparisons}
\section{Where do we go from here?}
\end{document}\bye




%\end{document}